\documentclass{article}

% Language setting
% Replace `english' with e.g. `spanish' to change the document language
\usepackage[english]{babel}

% Set page size and margins
% Replace `letterpaper' with`a4paper' for UK/EU standard size
\usepackage[letterpaper,top=2cm,bottom=2cm,left=3cm,right=3cm,marginparwidth=1.75cm]{geometry}

% Useful packages
\usepackage{amsmath}
\usepackage{graphicx}
\usepackage[colorlinks=true, allcolors=blue]{hyperref}

\title{Forgotten physics}
\author{You}

\begin{document}
\maketitle

\section{Classical mechanics}
\subsection{Quick recap}
\subsubsection{Lagrange}
With the Lagrange function $L=L(q,\dot q,t)=T-V$ 
\begin{align}
    S=\int dt L\quad\rightarrow\quad
    0=\delta S 
    &= \int dt\left[\frac{\partial L}{\partial q}\delta q+\frac{\partial L}{\partial \dot q}\delta\dot q\right]\\
    &= \int dt\left[\frac{\partial L}{\partial q}-\frac{d}{dt}\left(\frac{\partial L}{\partial \dot q}\right)\right]\delta q
\end{align}
The Euler-Lagrange equation look (obviously) a like Newtons third Law
\begin{align}
    \underbrace{\frac{\partial L}{\partial q}}_{\sim F=-\nabla V}-\frac{d}{dt}\underbrace{\left(\frac{\partial L}{\partial \dot q}\right)}_{\sim p} = 0.
\end{align}


\subsubsection{Hamilton}
We call
\begin{align}
    p=p(q,\dot q,t)=\frac{\partial L}{\partial \dot q}
\end{align}
the conjugated momentum. Now we can replace the generalized velocity with the conjugated momentum
\begin{align}
    H=H(q,p,t)
    &=\dot q\frac{\partial L}{\partial\dot q}-L\\
    &=p\dot q-L
\end{align}
Then
\begin{align}
    \delta L
    &=\frac{\partial L}{\partial q}\delta q+\frac{\partial L}{\partial \dot q}\delta\dot q\\
    &=\frac{\partial L}{\partial q}\delta q+p\delta\dot q\\
    &=\frac{\partial L}{\partial q}\delta q+\delta(p\dot q)-\dot q\,\delta p\\
    \delta H=\delta(p\dot q-L)&=\dot q\delta p-\frac{\partial L}{\partial q}\delta q=\dot q\delta p-\dot p\delta q\\
    \delta H&=\frac{\partial H}{\partial q}\delta q+\frac{\partial H}{\partial p}\delta p
\end{align}
which gives the Hamilton equations
\begin{align}
    \dot p_i=-\frac{\partial H}{\partial q_i},\quad\quad
    \dot q_i=\frac{\partial H}{\partial p_i}
\end{align}
For cyclical coordinates we have
\begin{align}
    \frac{\partial L}{\partial q_i}=0\quad\rightarrow\quad\frac{dp_i}{dt}=\frac{d}{dt}\frac{\partial L}{\partial\dot q_i}=0\quad\rightarrow\quad p_i=\alpha_i\\
    \dot q_i=\left.\frac{\partial H}{\partial p_i}\right|_{p_i=\alpha_i}=v_i(t)\quad\rightarrow\quad q_i=\int dt v_i+\beta_i
\end{align}



\subsection{Canonical transformations}



\end{document}